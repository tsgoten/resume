%%%%%%%%%%%%%%%%%%%%%%%%%%%%%%%%%%%%%%%%%%%%%%%%%%%%%%%%%%%%
% Original author: Ganesh Mohan (http://about.me/ganesh.mohan)
% author: Tarang Srivastava
%%%%%%%%%%%%%%%%%%%%%%%%%%%%%%%%%%%%%%%%%%%%%%%%%%%%%%%%%%%%

\documentclass{article}

\usepackage{fullpage}
\usepackage{amsmath}
\usepackage{amssymb}
\usepackage[T1]{fontenc}
\usepackage{fancyhdr}
\usepackage{lastpage}
\usepackage{graphicx}
\usepackage{hyperref}
\usepackage{fontawesome}
\usepackage{setspace}

\newcommand\blfootnote[1]{%
  \begingroup
  \renewcommand\thefootnote{}\footnote{#1}%
  \addtocounter{footnote}{-1}%
  \endgroup
}

% DEFINITIONS FOR RESUME

\textheight=10in
\pagestyle{fancy}
\raggedright
\fancyhf{}
\renewcommand{\headrulewidth}{0pt}

\setlength{\hoffset}{-2pt}
\setlength{\footskip}{20pt}

\def\bull{\vrule height 0.8ex width .7ex depth -.1ex }

\newcommand{\contact}[3]{
\vspace*{5pt}
\begin{center}
{\LARGE \scshape {#1}}\\
\vspace{3pt}
#2 
\vspace{2pt}
#3
\end{center}
\vspace*{-8pt}
}

\newcommand{\header}[1]{{
\hspace*{-15pt}\vspace*{6pt} \textsc{#1}} \vspace*{-6pt} 
\lineunder
}

\newcommand{\lineunder}{
\vspace*{-8pt} \\ \hspace*{-18pt} 
\hrulefill \\
}

\newcommand{\content}{
\vspace*{2pt}%
}

\newcommand{\school}[5]{\vspace*{2pt}% 
\textbf{#1} \labelitemi #2 \hfill #3 \\ #4 \hfill #5
\vspace*{5pt}
}

\newcommand{\college}[7]{
\textbf{#1} \labelitemi \textbf{#2} \hfill #3 \\ #4 \hfill #7 \\ #5 \\ #6 \vspace*{5pt}
}

\newcommand{\employer}[4]{{
\vspace*{2pt}%
\textbf{#1} #2 \hfill #3\\ #4 \vspace*{2pt}}
}

\newcommand{\project}[3]{{
\vspace*{2pt}% 
\textbf{#1} #2 \hfill #3\vspace*{2pt}}
}

\renewcommand{\labelitemi}{
$\vcenter{\hbox{\tiny$\bullet$}}$\hspace*{3pt}
}

\renewcommand{\labelitemii}{
$\vcenter{\hbox{\tiny$\bullet$}}$\hspace*{-3pt}
}

\newenvironment{bullet-list-major}{
\begin{list}{\labelitemii}{\setlength\leftmargin{3pt} 
\topsep 0pt \itemsep -2pt}}{\vspace*{4pt}\end{list}
}

\newenvironment{bullet-list-minor}{
\begin{list}{\labelitemii}{\setlength\leftmargin{15pt} 
\topsep 0pt \itemsep -2pt}}{\vspace*{4pt}\end{list}
}

\cfoot{
Last updated: July 7, 2018
}

\rfoot{
Page $\thepage\hspace*{3pt}\vert\hspace*{3pt}\pageref{LastPage}$
}

% END RESUME DEFINITIONS

\begin{document}

\small
\smallskip
\vspace*{-44pt}

\contact{\textbf{Tarang Srivastava}}
{\faPhone\ (609) 665-7567 
\labelitemi \faEnvelope\ \href{mailto:tarang.sriv@berkeley.edu}{tarang.sriv@berkeley.edu}
% \labelitemi \faLink\ \href{http://tarangsriv.me}{tarangsriv.me}
\labelitemi \faLinkedin\ \href{https://www.linkedin.com/in/tarangsriv/}{linkedin.com/in/tarangsriv}
\labelitemi \faGithub\ \href{https://github.com/tsgoten}{github.com/tsgoten}}%

\vspace{4pt}
\header{Education}
    \college{University of California, Berkeley}{Berkeley, CA}{August 2019 -- May 2022}
    {\textit{Double Major from the College of Letters and Sciences}}
    {\textit{Bachelor of Arts in \textbf{Computer Science}}}
    {\textit{Bachelor of Arts in \textbf{Applied Mathematics} \& Statistics Concentration}}
    {GPA: 3.85/4.0}

    \school{South Brunswick High School}{South Brunswick, NJ}{September 2015 -- June 2019}
    {\labelitemi Outstanding Math Student \labelitemi GSET nomination for Top 3 in class of 700 \labelitemi Honor Roll}
    {GPA: 4.4}

\vspace*{4pt}%
\header{Work Experience}

    \employer{Undergraduate Student Instructor (TA)}
    {-- UC Berkeley EECS Department}{June 2020 -- Present}{Berkeley, CA}
	\begin{bullet-list-minor}
	\item TA for (CS70) Discrete Mathematics and Probability Theory, a 700 person class covering topics like logic, graph theory, RSA, computability, discrete and continuous probability, Markov chains and more.
	\item Responsible for heading discussion sections, creating course material and administrating course grading.  
    \end{bullet-list-minor}

    % \employer{Computer Science Mentor (Tutor)}{-- UC Berkeley CSM}{Aug 2016 -- May 2019}{Berkeley, CA}
	% \begin{bullet-list-minor}
	% \item Take, make, cut, and deliver orders to customers.
	% \item Handle customer complaints, make end of the night deposits, cash register.
    % \end{bullet-list-minor}

    \employer{Software Engineering Intern}{-- Vydia}{June 2018 -- August 2018}{Holmdel, NJ}
	\begin{bullet-list-minor}
	\item Worked on developing and maintaining web and mobile applications for artists and producers.
	% \item Was part of a team of full-stack developers in Agile workflow. 
	\item Used JS, react and react-native for feature development in Agile workflow. Revamped DevOps with Docker. 
    \end{bullet-list-minor}

\vspace*{4pt}%
\header{Projects}

    \project{Math Animations}{ -- YouTube}{May 2020 - Present}
    \begin{bullet-list-minor}
    \item Using 3B1B's Manim tool to create math animations to describe and visualize intricate math topics.
    \item Videos include topics about polar decomposition and paradoxical probability topics. 
    \end{bullet-list-minor}

    \project{Recap}{ -- Stanford Hackathon}{February 2020}
    \begin{bullet-list-minor}
    \item Created a web application with React, to show local sourced news to a wider audience. 
    \item With a primary focus on creating an interface that incites discovery. 
    \end{bullet-list-minor}

    \project{Docker Migration}{ -- Vydia Internship}{ July 2018 -- August 2018}
	\begin{bullet-list-minor}
	\item Upgraded the entire teams development environment to use containerization tech. 
    \item Saves around four hours in onboarding process, and around half hour per developer in daily usage.
    \end{bullet-list-minor}

    \project{LED Authenticator}{ -- NJ Hackathon}{ May 2018}
	\begin{bullet-list-minor}
    \item Hardware device powered by Arduino and Android app to verify logins using AES. 
    \item Provides a two-step authenticator in a fun way, to encoursage usage. 
    \end{bullet-list-minor}

    \project{Gofer}{ -- Alpha Product}{ March 2018 - August 2018}
    \begin{bullet-list-minor}
    \item Android app for creating a network for local commerce by requesting and completing tasks.
    \item Needed a way to get stuff from around our town so we created this app to request help from seniors. 
    \end{bullet-list-minor}

    \project{Mousetrap/Electric Vehicle}{ -- Science Olympiad}{ January 2016 - March 2019}
    \begin{bullet-list-minor}
    \item Created the most efficient vehicle with stored potential energy in mousetrap. Placed 3rd in NJ. 
    \item Used Arduino to control vehicle to accurately drive a programmed distance. Placed 4th in NJ. 
    \end{bullet-list-minor}
	
\vspace*{4pt}%
\header{Technical Skills}
    \begin{bullet-list-major}
    \item Programming languages: \textbf{Python}, \textbf{Java}, JavaScript, Swift, Kotlin, Matlab, LaTeX, HTML/CSS
    \item Software: Docker, Android Studio, XCode, Git, Unity, Linux, SQL, Tensorflow 
    \item Non-Technical: Agile workflow, experience teaching students, organizing events of 50+ people.
    \end{bullet-list-major}

\vspace*{4pt}%
\header{Highlighted Coursework}
    \begin{bullet-list-major}
    \item \textbf{Computer Science:} Data Structures and Algorithms, Computer Programs, Android and iOS Development, Linux SysAdmin, Web Design, Designing Information Devices and Systems
    \item \textbf{Mathematics:} Real Analysis, Abstract Linear Algebra, Discrete Math, Probability Theory, Multivariable Calculus, Differential Equations, Numerical Analysis
    \end{bullet-list-major}

\vspace*{4pt}%
\header{Extra Curricular Activities}
    \begin{bullet-list-major}
    \item Volunteer to tutor group of 4-5 students learning discrete math and probability theory, CS 70 at Berkeley. 
    \item Social Chair in the Math Undergrad Student Association, hosting events to encourage math community.
    \item First bass in orchestra, wrestling tournament champion, and SciOly state medalist.
    \end{bullet-list-major}

\end{document}
